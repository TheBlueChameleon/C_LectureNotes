\chapter{Dateizugriff}
\epigraph{I have files, I have computer files and, you know, files on paper. But most of it is really in my head. So God help me if anything ever happens to my head!}{George R. R. Martin}

Ein- und Ausgabe fand bisher rein über den Bildschirm und die Tastatur statt. Selbstverständlich können wir aber auch Daten über die Festplatte oder andere Medien austauschen. Hier wollen wir Möglichkeiten kennenlernen, dies umzusetzen.

\section{File-Handle und Zugriffsmodi} \label{sec:fileAccess}
Um Dateien zu bearbeiten muss zunächst klar sein, welche Datei bearbeitet werden soll. Im Weiteren wird auch relevant, welcher Teil der Datei geschrieben oder gelesen werden soll, sowie eine ganze Reihe weiterer Informationen, die tief in die Organisation des Betriebssystems verwickelt sind. Für das Programmieren in C sind alle diese Informationen in einer \mintinline{c}{struct FILE} verpackt, die wir im Detail nicht kennen müssen\footnote{Unter \url{https://elixir.bootlin.com/linux/latest/source/include/linux/fs.h} finden Sie die Definition dieser Datenstruktur; der dort abgedruckte Header ist 3537 Zeilen lang, was die eigentliche Komplexität des Themas \emph{Dateizugriff} zeigt. Wie bereits erwähnt müssen wir zur Anwendung diese Details aber nicht kennen.}.

Dateizugriff kann in verschiedenen \emph{Modi} stattfinden. Modi sind in diesem Fall \emph{Lese-} oder \emph{Schreibzugriff} (sowie einige Unterarten davon, die wir gleich im Detail kennenlernen werden). Wir können also nicht im selben Schritt aus einer Datei lesen und in diese schreiben.

Wenn wir eine Datei bearbeiten wollen, erfragen wir zunächst beim Betriebssystem Zugriff auf diese Datei. Dies geschieht mit der Funktion \texttt{fopen}, welche im Header \texttt{<stdio.h>} deklariert wird. \texttt{fopen} erwartet als Parameter einen Dateinamen und eine Modusbezeichnung; zurück gegeben wird ein Pointer auf eine Instanz von \mintinline{c}{struct FILE}. Diesen Pointer bezeichnen wir im Weiteren als \emph{File-Handle}\footnote{\emph{Handle} ist hierbei bildlich zu verstehen: Es handelt sich um einen \enquote{Angriffspunkt}, ein \enquote{Griff}, an dem wir die Datei \enquote{anfassen} können.}.

\begin{codebox}[Syntax: \texttt{fopen}]
\begin{minted}{c}
FILE * handle = fopen(filename, mode);
\end{minted}
\end{codebox}
Dabei bedeuten:
\begin{description}
\item[\texttt{filename}] Ein \mintinline{c}{char *} auf einen beliebigen, gültigen\footnote{Wie Sie sicher wissen, sind bestimmte Zeichen in Dateinamen nicht erlaubt. Solche Zeichen sind beispielsweise \texttt{*} oder \texttt{?}} Dateinamen. Dieser \emph{String} darf auch Pfadangaben beinhalten\footnote{Unter unixoiden Systemen ist das Trenn-Symbol für Pfade der \emph{forward slash} \texttt{/}; für Windows wird der \emph{backward slash} \texttt{\textbackslash} verwendet}. Wird kein Pfad angegeben, so geht das System vom aktuellen Arbeitsverzeichnis aus\footnote{Dies ist im Allgemeinen dasselbe Verzeichnis, in dem auch die ausführbare Datei liegt. Befehle wie \texttt{system("cd ..");} können dieses Arbeitsverzeichnis natürlich wechseln.}.

\item[\texttt{mode}] Ebenfalls ein String (also ein \mintinline{c}{char *}), der angibt, ob die Datei im Lese- oder Schreibmodus geöffnet werden soll. Tabelle \ref{tab:FileModes} listet die wichtigsten Modi auf.
\end{description}

\begin{center}
\begin{tabular}{ll}
Modus & Wirkung \tabcrlf
\texttt {"r"}  & Lesezugriff (\emph{read})\\
\texttt {"w"}  & Schreibzugriff: Datei neu anlegen (\emph{write}) \\
\texttt{''a''} & Schreibzugriff: Daten an das Dateiende anhängen (\emph{append})
\end{tabular}
\captionof{table}{Liste der Dateimodi}\label{tab:FileModes}
\end{center}

\begin{codebox}[Beispiel: Datei \texttt{file.txt} im Schreibmodus öffnen (1)]
\begin{minted}[linenos]{c}
#include <stdio.h>

int main () {
  FILE * handle = fopen("file.txt", "w")
  return 0;
}
\end{minted}
\end{codebox}

Hier wird die Datei \texttt{file.txt} im \emph{Schreibmodus} geöffnet. Existiert die Datei noch nicht, so wird sie im Dateisystem angelegt. Wenn es dagegen schon eine Datei mit dem Namen \texttt{file.txt} gibt, so wird sie durch eine zunächst leere Datei ersetzt.

Kann die Datei nicht geöffnet werden (\eg weil der Dateiname unzulässige Zeichen enthält, oder weil der Dateipfad nicht erreichbar ist), so ist der Rückgabewert von \texttt{fopen} gleich 0.

Eine geöffnete Datei muss auch wieder geschlossen werden, und der Speicher für das File-Handle sollte wieder freigegeben werden. Beides geschieht in einem Schritt über den Befehl \texttt{fclose} (deklariert im Header \texttt{<stdio.h>}), der als Parameter lediglich den File-Handle erwartet:

\begin{codebox}[Syntax: \texttt{fclose}]
fclose(handle);
\end{codebox}

\begin{codebox}[Beispiel: Datei \texttt{file.txt} im Schreibmodus öffnen (2)]
\begin{minted}[linenos]{c}
#include <stdio.h>

int main () {
  FILE * handle = fopen("file.txt", "w")
  fclose(handle);
  return 0;
}
\end{minted}
\end{codebox}

Wenn Sie im obigen Beispiel in Zeile 4 den Modus \texttt{"w"} durch ein \texttt{''a''} ersetzen, wird eine bereits existierende Datei \texttt{file.txt} geöffnet und Daten können an das Dateiende angehängt werden. Sollte die Datei \texttt{file.txt} allerdings noch nicht existieren so wird diese neu erstellt und geöffnet.

Eine Datei kann nur dann im Lesemodus geöffnet werden, wenn diese auch tatsächlich existiert. Ersetzen wir in Zeile 4 also das \texttt{"w"} durch ein \texttt{"r"}, so wird nur dann ein
\mintinline{c}{struct FILE} angelegt, wenn das Öffnen erfolgreich war. Andernfalls ist der Rückgabewert von \texttt{fopen} gleich 0. Im Lesemodus wird keine Datei angelegt.

\begin{hintbox}[Prüfen ob Datei bereits vorhanden]
Es gibt Situationen, in denen Sie nur prüfen wollen, ob eine Datei mit einem gegebenen Namen überhaupt existiert. Dies können Sie über den Rückgabewert von \texttt{fopen} leicht prüfen:

\begin{codebox}[Beispiel: Prüfen ob Datei bereits vorhanden]
\begin{minted}[linenos]{c}
#include <stdio.h>

int fileExists(char * filename) {
  FILE * handle = fopen(filename, "r");
  if (handle) {fclose(file); return 1;}
  else        {              return 0;}
}

int main () {
  FILE * handle = NULL;
  if (fileExists("myFile.txt")) {
    // Dialog anzeigen: Datei überschreiben?
  } else {
    handle = fopen("myFile.txt", "w");
  }

  if (handle) {fclose(handle);}
  return 0;
}
\end{minted}
\end{codebox}
\end{hintbox}

\begin{hintbox}[Daten nicht versehentlich überschreiben]
Öffnen Sie eine Datei die im Modus \texttt{"w"}, so wird entweder eine leere Datei angelegt, oder eine bestehende Datei durch eine leere Datei ersetzt. Wollen Sie versehentliches Überschreiben verhindern, so können Sie als Modus auch \texttt{"wx"} angeben: dies führt automatisch eine Prüfung durch, ob eine Datei mit gegebenem Namen bereits existiert und bricht den Vorgang ab, wenn dies der Fall ist. Als Handle wird in diesem Fall NULL zurück gegeben:

\begin{codebox}[Beispiel: Datei im Schreibmodus öffnen ohne versehentliches Überschreiben]
\begin{minted}[linenos]{c}
#include <stdio.h>

int main () {
  FILE * handle = fopen("myFile.txt", "wx");
  if (handle) {
    // Datei schreiben...
  } else {
    printf("Fehler: Datei existiert bereits!\n");
    return -1;
  }

  if (handle) {fclose(handle);}
  return 0;
}
\end{minted}
\end{codebox}
\end{hintbox}

\section{Text-Zugriff -- Lesen und Schreiben in menschlichem Format} \label{sec:humanAccess}
In die mit \texttt{fopen} geöffnete Datei können wir nun mit \texttt{fprintf} schreiben, ganz so wie wir es schon von \texttt{printf} gewohnt sind. Als zusätzlichen, ersten Parameter müssen wir angeben, in welche Datei geschrieben werden soll: Dies geschieht selbstverständlich über das File-Handle:

\begin{codebox}[Syntax: \texttt{fprintf}]
fprintf(handle, formatString, ...)
\end{codebox}

\begin{codebox}[Beispiel: Ausgabe in eine Datei]
\begin{minted}[linenos]{c}
#include <stdio.h>

int main () {
  FILE * handle = fopen("myFile.txt", "w");

  fprintf(handle, "hello world!\n");
  fprintf(handle, "29 * 3 = %d\n", 29 * 3);

  fclose(handle);
  return 0;
}
\end{minted}
\end{codebox}

Dieses Codebeispiel sorgt dafür, dass nach Programmablauf im Arbeitsverzeichnis eine Datei mit Namen \texttt{myFile.txt} vorliegt. Der Inhalt dieser Datei lautet:

\begin{cmdbox}%
	[Inhalt der Datei \texttt{myFile.txt} nach Ablauf des Beispiels: \emph{Ausgabe in eine Datei}]
hello world!\\
29 * 3 = 87\\

\end{cmdbox}
(Der letzte Zeilenumbruch im obigen Abdruck ist kein Versehen -- die Datei hat drei Zeilen, wobei die letzte eine \enquote{leere Zeile} ist.)

Dieselbe Datei \texttt{myFile.txt} kann nun mit dem Befehl \texttt{fscanf} wieder gelesen werden. Ähnlich wie \texttt{fprintf} ist dieser Befehl ein Analogon zu \texttt{scanf} für Dateien: In beiden Fällen steht das erste \texttt{f} für \emph{file}.

\begin{codebox}[Syntax: \texttt{fscanf}]
fscanf(handle, formatString, pointer1, ...);
\end{codebox}

Aus einer zuvor geöffneten Datei mit \texttt{handle} lesen wir also nacheinander Werte, wie sie ein \texttt{formatString} vorgibt. Dabei können wir uns vorstellen, wie ein Cursor in der Datei langsam nach vorne wandert, und für jedes \texttt{\%} im \texttt{formatString} einen Informationsblock weiter springt. Die gelesenen Daten werden dann wie bei \texttt{scanf} an jenen Speicheradressen abgelegt, auf welche die \texttt{pointer} zeigen.

Während die Syntax für uns inzwischen leicht zu verstehen ist, gestaltet sich die Anwendung manchmal schwierig, wie die folgenden Beispiele verdeutlichen:

\begin{codebox}[Beispiel: Datei \texttt{myFile.txt} zurück lesen]
\begin{minted}[linenos]{c}
#include <stdio.h>

int main () {
  FILE * handle = fopen("myFile.txt", "r");

  if (!handle) {
    printf("Fehler: Konnte Datei 'myFile.txt' nicht öffnen.");
    return -1;
  }

  char   helloWordLine[1024],
         asterisk     [2],
         equals       [2];
  int    twentynine = -1, three = -1, eightyseven = -1;

  fscanf(handle, "%1023[^\n]",  helloWordLine);
  fscanf(handle, "%d"        , &twentynine);
  fscanf(handle, "%1s"       ,  asterisk);
  fscanf(handle, "%d"        , &three);
  fscanf(handle, "%1s"       ,  equals);
  fscanf(handle, "%d"        , &eightyseven);

  printf("%s\n", helloWordLine);
  printf("%d %s %d %s %d\n",
    twentynine, asterisk, three, equals, eightyseven
  );

  fclose(handle);
  return 0;
}
\end{minted}
\end{codebox}

\begin{cmdbox}[Ausgabebeispiel: Datei \texttt{myFile.txt} zurück lesen]
hello world!\\
29 * 3 = 87
\end{cmdbox}

Wie bekannt öffnen wir in Zeile 4 unsere Datei und überprüfen anschließend (Zeile 6) ob das erfolgreich war.

In den Zeilen 10 bis 13 deklarieren wir einige Variablen, in denen wir den Dateiinhalt speichern wollen. \texttt{helloWorldLine} ist sehr groß gewählt, um mit Sicherheit eine ganze Zeile speichern zu können. \texttt{asterisk} und \texttt{equals} enthalten jeweils ein Zeichen und einen Null-char. Die anderen Variablen werden schließlich die Zahl aufnehmen, nach der sie benannt sind.

In Zeile 15 lesen wir die erste Zeile der Datei ein. Der Format-String
\texttt{\%1023[\textasciicircum\textbackslash n]}
übersetzt sich zu \enquote{maximal 1023 Zeichen, die alles außer ein Zeilenumbruch sein dürfen}. Wie immer wird bei String-Lesebefehlen ein Null-char mit geschrieben, daher dürfen nur 1023 Zeichen gelesen werden, auch wenn \texttt{helloWorldLine} 1024 Zeichen aufnehmen kann. Beim Lesen der ersten Zeile wird durch den Formatstring der Zeilenumbruch nicht mit eingelesen; der Cursor steht also noch \emph{vor} dem Zeilenumbruch.

In der folgenden Zeile 16 soll nun eine Ganzzahl gelesen werden (Formatstring \texttt{\%d}). Daher überspringt \texttt{fscanf} alle Nicht-Zahlen-Zeichen. In unserem Fall ist das also der Zeilenumbruch. Danach trifft der Dateicursor auf den Text \texttt{29}, der korrekt als Zahl eingelesen wird.

Als nächstes soll das Sternchen \texttt{*} gelesen werden. Zeile 17 liest genau dieses eine Zeichen ein, und speichert es unter der Adresse von \texttt{asterisk}. Wir benutzen dazu den Format-String
\texttt{\%1s}, der eben einen String mit einer Länge von einem Zeichen beschreibt, wobei Zeilenumbrüche, Leerzeichen und Tabulatoren ignoriert werden.

Die Idee liegt nahe, nur ein \mintinline{c}{char} lesen zu wollen, da ja nur ein einzelnes Sternchen gelesen werden soll. Hierbei würde man aber vergessen, dass vor diesem Zeichen noch ein Leerzeichen steht. Ersetzen wir Zeile 17 durch \mintinline{c}{fscanf(handle, "%c", asterisk);}, so enthält \texttt{*asterisk} jetzt den Wert \texttt{32}, also den ASCII-Code eines Leerzeichens. Der Dateicursor ist nur um eine Stelle weiter gewandert, und steht weiterhin \emph{vor} dem Sternchen, weswegen hier auch das Einlesen der nächsten Zahl in Zeile 18 fehlschlägt.

In den folgenden Zeilen 18-20 wird dieses Pattern wiederholt. Schließlich erfolgt in den Zeilen 22-25 die Ausgabe der eingelesenen Werte mit \texttt{printf}, bevor Zeile 27 die Datei schließt und Zeile 28 das Programm beendet.

\section{Binär-Zugriff -- Lesen und Schreiben in Binärformat} \label{sec:binAccess}
Der Titel des vorherigen Abschnitts war \emph{Lesen und Schreiben in menschlichem Format}. Alle Werte, die in die Datei geschrieben wurden, werden von jedem Texteditor so dargestellt, dass ein Mensch diese interpretieren kann. Sie sind inzwischen aber mit der Tatsache vertraut, dass die interne Speicherdarstellung von diesem Menschenlesbaren Format stark abweicht.

Es ist möglich, Werte so in Dateien abzulegen, wie sie auch im Speicher stehen. Dies hat zwei Vorteile:
\begin{itemize}
\item Da weder eine Umrechnung vom menschenlesbaren Format in die interne Speicherstruktur stattfinden
		muss, noch auf Trennzeichen wie Zeilenumbrüche acht gegeben werden muss, funktioniert das
		Einlesen solcher \emph{Binärdaten} bedeutend schneller. Für große Datensätze (\eg Bilder,
		wissenschaftliche Messreihen, Musikdaten, \ldots) bedeutet das erheblichen Performance-Gewinn.
\item Im Allgemeinen ist die Binäre Darstellung von Zahlen kompakter. Die Zahl \texttt{12345678}
		braucht als menschenlesbarer Text 8 Zeichen und damit auch 8 Bytes. Hinzu kommt noch
		mindestens ein Byte für ein Trennzeichen. Ein \mintinline{c}{long}-Wert dagegen ist immer 4
		Byte lang, und kann daher ohne Trennzeichen abgelegt werden. Im Binärformat gespeicherte Dateien
		benötigen also im Allgemeinen weniger Speicherplatz.
\end{itemize}

Zu diesem Zweck existieren die Befehle \texttt{fread} und \texttt{fwrite}, beide deklariert im Header \texttt{<stdio.h>}\footnote{TODO: Endianness}.

Da die binäre Darstellung besonders für große Datenmengen günstig ist, sind die Befehle \texttt{fread} und \texttt{fwrite} auf die Benutzung mit Arrays ausgelegt. Die Syntax der beiden Befehle ist sehr ähnlich:
\begin{codebox}[Syntax: \texttt{fread} und \texttt{fwrite}]
fread (pointer, size, count, handle);\\
fwrite(pointer, size, count, handle);
\end{codebox}
Dabei stehen die Parameter für folgende Konzepte:
\begin{description}
\item[\texttt{pointer}] Die Speicheradresse, an der die gelesenen Werte abgelegt werden sollen
		(\texttt{fread}) bzw. von der die Daten stammen, die in die Datei geschrieben werden sollen
		(\texttt{fwrite}).
\item[\texttt{size}] Die Größe eines zu lesenden/schreibenden Elements in Bytes.
\item[\texttt{count}] Die Anzahl der Elemente, die gelesen/geschrieben werden sollen.
\item[\texttt{handle}] Ein File-Handle, wie er von \texttt{fopen} zurückgegeben wurde.
\end{description}

Betrachten wir dazu das folgende Beispiel:

\begin{codebox}[Beispiel: \texttt{fwrite} und \texttt{fread}]
\begin{minted}[linenos]{c}
#include <stdio.h>

int main () {
  FILE * handle;
  handle = fopen("myFile.dat", "w");
  if (!handle) {
    printf("Fehler: Konnte Datei 'myFile.dat' nicht öffnen.");
    return -1;
  }

  unsigned int outArray[26];
  unsigned int N = sizeof(outArray) / sizeof(*outArray);

  for (int i = 0; i < N; i++) {
    // einige große Zahlen erzeugen
    outArray[i] =   'A' + i               // =  65 + i
                + (('A' + i) <<  8)       // = (65 + i) *      256
                + (('A' + i) << 16)       // = (65 + i) *    65536
                + (('A' + i) << 24);      // = (65 + i) * 16777216
  }
  fwrite(outArray, N, sizeof(*outArray), handle);

  fclose(handle);
  // ....................................................................... //
  handle = fopen("myFile.dat", "r");
  if (!handle) {
    printf("Fehler: Konnte Datei 'myFile.dat' nicht öffnen.");
    return -1;
  }

  unsigned int inArray[N+1] = {0};
  fread (inArray, N, sizeof(* inArray), handle);

  for (int i = 0; i < N; i++) {
    printf("%2d\t%u\t%u\t%s\n",
           i,
           inArray[i] ,  outArray[i],
           inArray[i] == outArray[i]  ?  "gleich"  : "ungleich"
          );
  }

  printf("\nDarstellung derselben Daten in der Datei:\n");
  printf("%s\n", (char *)inArray);

  fclose(handle);
  return 0;
}
\end{minted}
\end{codebox}

Wie gewohnt öffnen wir in Zeile 6 eine Datei im Schreibmodus und überprüfen anschließend, ob dies erfolgreich war. Zeile 12 deklariert ein Array \texttt{outArray} aus 26 positiven Ganzzahlen, die wir später in die Datei \texttt{myFile.dat} schreiben werden. Die Variable \texttt{N} enthält die Anzahl der Elemente von \texttt{outArray}. (Obwohl wir wissen, dass dieses Array 26 Elemente hat, bietet es sich manchmal an, die Arraygröße explizit zu berechnen. Wenn wir zu späterer Zeit unseren Code anpassen und die Arraygröße ändern, müssen wir nicht den ganzen Code überarbeiten.)

Wir befüllen \texttt{outArray} in den Zeilen 15 bis 21 mit einigen großen Zahlen. Für dieses Beispiel sind diese Zahlen so konstruiert, dass ihre Speicherdarstellung dieselbe Form hat wie ein String aus den Buchstaben A-Z. Vorerst ist es aber egal, welche Zahlen in diesem Array stehen.

Zeile 23 schreibt dann dieses Array \emph{als Ganzes} in die Datei \texttt{myFile.dat}. Wie bereits erwähnt werden die Daten in \emph{binärer Form} geschrieben, \ie ein Mensch wird in dieser Datei keine Zahlen erkennen, wenn diese mit einem Texteditor geöffnet wird. Entsprechend können wir den Inhalt auch nicht mit \texttt{fscanf} zurück lesen, sondern müssen mit \texttt{fread} arbeiten.

Um die Daten zurück zu lesen, muss die Datei \texttt{myFile.dat} zuerst geschlossen werden, da wir sie bis dato im \emph{Schreibmodus} geöffnet hatten. Wir setzen daher also in Zeile 25 ein \texttt{fclose} und öffnen die Datei erneut in Zeile 29, diesmal im Lesemodus.

Zeile 35 deklariert ein neues Array \texttt{inArray}, das wir nun um ein Element größer wählen, als es für unsere Datei nötig wäre. Dies dient hier dem Zweck, die eingelesenen Daten auch als String ausgeben zu können -- siehe dazu gleich mehr.

Ebenso, wie wir ein gesamtes Array in einem einzigen Befehl schreiben konnten, ist es in Zeile 36 auch möglich, eine gesamte Liste mit einem einzigen \texttt{fread} zu lesen. Die ersten 26 Elemente des Arrays enthalten jetzt den Dateiinhalt; das \enquote{überzählige} 27. Element wird nicht überschrieben und behält seinen Startwert 0.

Die Zeilen 38-44 geben sowohl \texttt{inArray} als auch \texttt{outArray} aus, sowie einen Vergleich der gelesenen und geschriebenen Werte. Wie zu erwarten sind alle Array-Elemente von \texttt{inArray} und \texttt{outArray} gleich.

In Zeile 47 schließlich geben wir nochmals den Inhalt von \texttt{inArray} auf dem Bildschirm aus, interpretieren ihn hier jedoch als Zeichenkette. Dies ist auch der Grund, warum wir ein überzähliges Element für das Feld angelegt haben: das letzte Element des Arrays hat den Wert \texttt{0} und markiert so das Ende eines Strings. Diesen String werden Sie auch sehen, wenn Sie die Datei \texttt{myFile.dat} mit einem Texteditor öffnen!

Alle Daten, mit denen Sie arbeiten, können auf verschiedene Art und Weise interpretiert werden. Obwohl Ihre Arrays \texttt{inArray} und \texttt{outArray} positive Ganzzahlen enthalten, werden diese im Speicher wie alles als Bitmuster behalten, die in Dateien wie ein Text aussehen können!

\begin{cmdbox}[Ausgabebeispiel: \texttt{fwrite} und \texttt{fread}]
\begin{minted}{text}
 0      1094795585      1094795585      gleich
 1      1111638594      1111638594      gleich
 2      1128481603      1128481603      gleich
 3      1145324612      1145324612      gleich
 4      1162167621      1162167621      gleich
 5      1179010630      1179010630      gleich
 6      1195853639      1195853639      gleich
 7      1212696648      1212696648      gleich
 8      1229539657      1229539657      gleich
 9      1246382666      1246382666      gleich
10      1263225675      1263225675      gleich
\end{minted}
\end{cmdbox}
%
\begin{cmdbox}[]
\begin{minted}{text}
11      1280068684      1280068684      gleich
12      1296911693      1296911693      gleich
13      1313754702      1313754702      gleich
14      1330597711      1330597711      gleich
15      1347440720      1347440720      gleich
16      1364283729      1364283729      gleich
17      1381126738      1381126738      gleich
18      1397969747      1397969747      gleich
19      1414812756      1414812756      gleich
20      1431655765      1431655765      gleich
21      1448498774      1448498774      gleich
22      1465341783      1465341783      gleich
23      1482184792      1482184792      gleich
24      1499027801      1499027801      gleich
25      1515870810      1515870810      gleich

Darstellung derselben Daten in der Datei:
AAAABBBBCCCCDDDDEEEEFFFFGGGGHHHHIIIIJJJJKKKKLLLLMMMMNNNNOOOOPPPPQQQQRRRRSSSSTTT
TUUUUVVVVWWWWXXXXYYYYZZZZ
\end{minted}
\end{cmdbox}

\section{Cursorposition in Dateien: \texttt{fseek} und \texttt{ftell}}
Nach dem Öffnen einer Datei befindet sich der Cursor entweder am Dateianfang (Modi \texttt{r} und \texttt{w}) oder am Dateiende (Modus \texttt{a}). Nachfolgende Lese- und Schreibbefehle bewegen den Cursor. Wir können aber auch die Cursorposition frei wählen, ohne erst elementweise durch die Datei zu springen. Dazu dient der Befehl \texttt{fseek}.

\begin{codebox}[Syntax: \texttt{fseek}]
fseek(filehandle, offset, origin);
\end{codebox}

Dabei bedeuten:
\begin{description}
\item[\texttt{filehandle}] Das Handle, wie wir es von \texttt{fopen} erhalten haben.
\item[\texttt{offset}] Die Zahl von Bytes (\ie Schriftzeichen), um die der Cursor verschoben werden soll.
\item[\texttt{origin}] Eine Codeziffer, die angibt, von wo weg gezählt werden soll. Mögliche Werte sind:
	\begin{description}
	\item[\texttt{SEEK\_SET}] Verschiebung relativ zum Dateianfang. (\texttt{offset} ist dann die Nummer
		 des Bytes in der Datei.)
	\item[\texttt{SEEK\_CUR}] Verschiebung relativ zur aktuellen Cursorposition
	\item[\texttt{SEEK\_END}] Verschiebung relativ zum Dateiende.
	\end{description}
\end{description}

Analog dazu existiert der Befehl \texttt{ftell}, der die aktuelle Cursorposition in der Datei zurück gibt:
\begin{codebox}[Syntax: \texttt{ftell}]
long int position = ftell(filehandle);
\end{codebox}

Mit diesen beiden Befehlen können wir beispielsweise die Größe einer Datei in Bytes ermitteln:

\begin{codebox}[Beispiel: Länge einer Datei mit \texttt{fseek} und \texttt{ftell} ermitteln]
\begin{minted}[linenos]{c}
#include <stdio.h>

long int lenght_of_file (FILE * filehandle) {
  long int oldPos = ftell(filehandle);

  fseek(filehandle, 0L, SEEK_END);
  long int endPos = ftell(filehandle);

  fseek(filehandle, oldPos, SEEK_SET);
  return endPos;
}

int main () {
  FILE * filehandle = fopen("file.txt", "r");
  printf(
    "Die Datei 'file.txt' ist %ld Bytes groß.\n",
    lenght_of_file(filehandle)
  );
  fclose(filehandle);
}
\end{minted}
\end{codebox}

Wir speichern zuerst die aktuelle Position in der Datei (Zeile 4). Danach springen wir \enquote{0 Bytes vor das Dateiende} (Zeile 6). Wir ermitteln nun die Position, die ja auch zugleich die Dateilänge angibt (Zeile 7). Schließlich springen wir an den Ausgangspunkt in der Datei zurück, also an die Position, die wir zu Anfang gespeichert hatten (Zeile 9).

Schließlich gibt es noch einen schnellen Weg, zu ermitteln, ob wir bereits das Dateiende erreicht haben: Die Funktion \texttt{feof} (deklariert im Header \texttt{<stdio.h>}; steht für \emph{file operation: End of File}) nimmt als Argument ein File-Handle, und gibt \texttt{1} zurück, sobald das Dateiende erreicht wurde:

\begin{codebox}[Beispiel: Datei bis ans Ende lesen mit \texttt{feof}]
\begin{minted}[linenos]{c}
#include <stdio.h>

int main () {
  FILE * filehandle = fopen("file.txt", "r");

  char line[1024];
  while (!feof(filehandle)) {
    fscanf(filehandle, "%1023[^\n]", line);
    printf("%s\n", line);
  }

  fclose(filehandle);
}
\end{minted}
\end{codebox}
