\section*{Vorwort}
Dieses Script soll Sie in der Vorlesung \emph{\myTitle} im \currentPeriod begleiten. Alle wichtigen Kursinhalte werden hier behandelt und anhand von Beispielen verdeutlicht. Es versteht sich von selbst, dass ein optimaler Lernerfolg nur bei Besuch der Vorlesung gegeben ist. Vorkenntnisse in anderen Programmiersprachen sind zur Arbeit mit diesem Script nicht notwendig.

Hier wird von einer Linux-Arbeits\-umgebung ausgegangen und minimale Grundkenntnisse dieser Arbeits\-umgebung vorausgesetzt (Starten einer Kommandozeilen-Umgebung, Wechsel des Arbeitsverzeichnisses, Aufrufen von Programmen aus der Kommandozeile, Übergabe von Parametern an Programme). Die DozentInnen des Kurses können bei Bedarf erklären, wie die Arbeitsumgebung bedient wird.

Sie werden primär die Grundlagen der Programmiersprache C (Standardisierung der Sprache von 2011) erlernen. Die Sprache C erschien im Jahr 1972 und wird seither kontinuierlich weiterentwickelt. C erlaubt sehr systemnahe und daher effiziente und vielseitige Programmierung, was die lange Lebensdauer der Sprache erklärt. Anwendungsgebiete umfassen \ua wissenschaftliches Rechnen, Kernels von Betriebssystemen oder Gerätetreiber.

C inspirierte die Entwicklung vieler anderer Programmiersprachen (darunter C++, C\#, Java, D, Go, \ldots). Die hier erworbenen Kenntnisse sind daher leicht in andere Arbeitsfelder übertragbar und sind daher für Sie ein Sprungbrett in die Arbeit als ProgrammiererIn.

Die Sprache C++ versteht sich als Weiterentwicklung der Sprache C und soll hier nur in Ausblicken behandelt werden. Während C und C++ syntaktisch sehr ähnlich sind, handelt es sich um eigenständige Sprachen, die getrennt voneinander behandelt werden sollten. In diesem Sinne bereitet Sie dieser Kurs darauf vor, einen \enquote{vollen} C++-Kurs zu besuchen, wie er an der Universität Regensburg zum Ende jedes Semesters als Blockkurs angeboten wird.

\begin{plusbox}[C++-spezifische Inhalte]
Inhalte, die sich auf C++ beziehen sind optisch durch Boxen vom restlichen Kurs abgehoben.
\end{plusbox}

Dieses Dokument ist keine vollständige Referenz der Sprachen C oder C++. Hier sollen nur die Grundlagen der Sprache C sowie Ausblicke auf die Arbeit mit C++ gegeben werden. Als Befehlsreferenz beider Sprachen empfehle ich auf die Seiten \url{https://en.cppreference.com/w/} (englische, ausführlichere Version) und bzw. \url{https://de.cppreference.com/w/} (deutsche Version mit eingeschränkter Verfügbarkeit der Artikel), auf die auch im Rahmen dieses Scriptes gelegentlich verwiesen wird.

Dieses Script wurde nach bestem Wissen und Gewissen zusammengestellt; Code-Beispiele wurden auf wenigstens einer Maschine getestet. Dennoch können menschliche Fehler nicht ausgeschlossen werden. Wem Unstimmigkeiten auffallen oder wer Vorschläge und Anregungen zu diesem Text einbringen will, möge mir dies sehr gerne mitteilen. Ich bin erreichbar unter der Email-Adresse:\\ \url{stefan.hartinger@stud.uni-regensburg.de}.
\begin{flushright}
\myName, \myVersionTime
\end{flushright}

%Für den Blockkurs im Sommersemester 2021 ist Nils Meyer der Ansprechpartner. Email-Adresse: \url{nils.meyer@ur.de}. Am besten schreibt ihr eine Mail an uns beide.
%\begin{flushright}
%Nils Meyer, März 2021
%\end{flushright}

\vfill

%\section*{Danksagung}
%Ich danke Stefan Solbrig und
