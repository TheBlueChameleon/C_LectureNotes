\chapter{Anbindung an das Betriebssystem} \label{chp:OS-Link}
\epigraph{Operating Systems are like underwear -- nobody really wants to look at them}{Bill Joy}

Während unsere Programme \enquote{für sich alleine stehen können}, werden sie in den meisten Fällen auf einem Rechner ausgeführt, der ein Betriebssystem zur Verfügung stellt. Dieses stellt einige Schnittstellen zur Verfügung, die wir hier kennenlernen wollen.

\section{Errorcodes} \label{sec:errorcodes}
Ein Errorcode ist ein Wert zwischen 0 und 255, der angibt, welcher Fehler bei der Ausführung aufgetreten ist. Lief das Programm ohne Fehler ab, so ist dieser Errorcode üblicherweise gleich \texttt{0}.

\subsection{Rückgabewert der \texttt{main}}
In Kapitel \ref{chp:funcs} haben wir bereits festgestellt, dass formal die \mintinline{c}{int main} eine Funktion ist, die einen \mintinline{c}{int} zurück gibt. Bisher haben wir von dieser Funktionalität keinen Gebrauch gemacht. Ab hier wollen wir aber dieses Feature mit nutzen, und beenden unsere Programme mit einer \mintinline{c}{return}-Anweisung:
\begin{codebox}[Beispiel: Hello World mit Rückgabewert]
\begin{minted}[linenos]{c}
#include <stdio.h>

int main() {
  printf("Hello World!\n");
  return 0;
}
\end{minted}
\end{codebox}

Zeile 5 beendet das Programm und gibt einen Wert \texttt{0} zurück -- aber wo wird dieser Wert entgegen genommen? Wie Sie aus dem Thema dieses Kapitels sicher schon erahnen, nimmt das Betriebssystem den Rückgabewert entgegen. Im Allgemeinen wird dort der Wert ignoriert; er kann aber wiederum an andere Programme weiter gereicht werden, die so auf den Ablauf unseres Programms reagieren. Üblicherweise wird ein \emph{Errorcode} zurück gegeben.

Eine Möglichkeit, den Rückgabewert eines Programms abzufragen ist, in der Konsole das Symbol \texttt{\$?} zu benuzen. Geben Sie die Zeile
\begin{cmdbox}[Kommandozeile: Rückgabewert eines Programms anzeigen]
echo \$?
\end{cmdbox}
ein. Sie werden eine Zahl sehen, die eben genau den Rückgabewert des zuletzt gestarteten Programms beschreibt.

Beispiel: Der Linux-Befehl \texttt{ls} listet den Inhalt des aktuellen Arbeitsverzeichnisses auf. \texttt{progDoesNotExist.c} sei der Name einer Datei, die nicht existiert. Mit diesem Wissen können sie den folgenden Konsolen-Output verstehen:

\begin{cmdbox}[Beispiel: Ausgabe von Rückgabewerten verschiedener Programme auf der Konsole]
\begin{minted}{text}
blue-chameleon@blue-chameleon:~/Codes$ ls
 a.out                        gtk_helloworld.c     playground.c
 BosonScattering              ImageLibrary         ProcessFilesVBA.bas
 CasseBriques                 Java                 project
 CB                           make_play            QS_10-Hartinger_backup.c
'ChiZeta Default CTors.cpp'   Mockepon             QS_10-Hartinger.c
 CircleAreaApprox.c           Mockepon_old_stuff   tmp
 debug-out.c                  module.c             vector2D.c
 definitions.h                myProgram            vector2D.h
 findroot_sqrt.c              myProgram.c          VivadoProjects
 freeglut_test.c              Playground
blue-chameleon@blue-chameleon:~/Codes$ echo $?
0
blue-chameleon@blue-chameleon:~/Codes$ gcc progDoesNotExist.c
gcc: error: progDoesNotExist.c: Datei oder Verzeichnis nicht gefunden
gcc: fatal error: no input files
compilation terminated.
blue-chameleon@blue-chameleon:~/Codes$ echo $?
1
\end{minted}
\end{cmdbox}
\texttt{ls} listet den Inhalt des aktuellen Arbeitsverzeichnisses auf und wird ohne Fehler beendet. Entsprechend wird hier der Errorcode 0 gesetzt. Da die Datei \texttt{progDoesNotExist.c} nicht existiert, bricht der gcc seinen Prozess mit Fehler ab; wir erhalten den Error Code 1.

Errorcodes sind teilweise normiert. Das bedeutet, dass die Bedeutung einiger Fehlercodes festgelegt ist; ab einem bestimmten Code sind diese Werte aber frei zu vergeben, \ie der/die ProgrammiererIn kann selbst entscheiden, was dieser Wert bedeuten soll.

Für die vordefinierten Fehlercodes sind im Header \texttt{<errno.h>} bereits Symbole angelegt. Siehe \url{http://man7.org/linux/man-pages/man3/errno.3.html} für eine Liste aller Symbole mit zugeordneter Bedeutung. Unter \url{http://www-numi.fnal.gov/offline_software/srt_public_context/WebDocs/Errors/unix_system_errors.html} finden Sie auch die Zuordnung der Symbole zu den Zahlen.

Selbst definierte Errorcodes sollten größer als 130 sein, um keine Bedeutungskollision mit den vordefinierten Codes zu ergeben.

\begin{hintbox}[Allgemeiner Errorcode \texttt{-1}]
In den meisten Anwendungsfällen reicht es, zu unterscheiden: \emph{Programm wurde erfolgreich abgeschlossen} und \emph{ein Fehler ist aufgetreten}. Wo diese Unterscheidung reicht, wird als Errorcode entweder \texttt{0} bei Erfolg und \texttt{-1} bei Fehler zurück gegeben. Vom Betriebssystem wird dies automatisch in den Errorcode \texttt{255} übersetzt.
\end{hintbox}

Im Kurs \emph{Linux} der Universität Regensburg lernen Sie Möglichkeiten kennen, diese Errorcodes auszuwerten.

\subsection{Programm jederzeit beenden: \texttt{exit}}
Neben der Möglichkeit, die \texttt{main} mit \texttt{return errorcode} zu verlassen, ist es auch möglich, den Befehl \texttt{exit} aus der \texttt{<stdlib.h>} zu benutzen. Dieser beendet die Programmausführung \emph{von jedem beliebigen Punkt aus}, also auch, wenn gerade eine Funktion ausgeführt wird. Der Befehl \texttt{exit} nimmt als Argument den Errorcode an, der an das Betriebssystem zurück gegeben werden soll.

\begin{codebox}[Beispiel: Programm beenden mit \texttt{exit}]
\begin{minted}[linenos]{c}
#include <stdio.h>
#include <stdlib.h>
#include <math.h>
 
int main() {
  double number;
  
  printf("Bitte geben Sie eine positive Zahl ein:");
  scanf("%lf", &number);
  
  if (number < 0) {
    printf("Fehler: Zahl war kleiner als 0!\n");
    exit(EXIT_FAILURE);  // Fehlercode -1
  }
 
  printf("Die Wurzel der Eingabe war %lf.\n", sqrt(number));
  return EXIT_SUCCESS;   // Fehlercode 0
}
\end{minted}
\end{codebox}

\subsection{Befehle an die Kommandozeile: \texttt{system}}
Der Befehl \texttt{system} aus der \texttt{<stdlib.h>} startet ein externes Programm und gibt den Errorcode dieses Aufrufs zurück. Wir können das Beispiel: \emph{Ausgabe von Rückgabewerten verschiedener Programme auf der Konsole} also auch direkt aus unserem C-Code heraus laufen lassen:

\begin{codebox}[Beispiel: Aufrufe mit \texttt{system} und Auswertung der Errorcodes]
\begin{minted}[linenos]{c}
#include <stdio.h>
#include <stdlib.h>
 
int main() {
  printf("Inhalt des aktuellen Arbeitsverzeichnisses:");
  int errCode_ls = system("ls");
  printf("ls schließt ab mit Errorlevel %d\n", errCode_ls);

  printf("Versuch der Kompilation von 'progDoesNotExist.c':");
  int errCode_pdne = system("gcc progDoesNotExist.c");
  printf("Errorlevel %d\n", errCode_pdne);
  
  return 0;
}
\end{minted}
\end{codebox}

\begin{cmdbox}[Ausgabebeispiel: Aufrufe mit \texttt{system} und Auswertung der Errorcodes]
\begin{minted}{text}
 a.out                        gtk_helloworld.c     playground.c
 BosonScattering              ImageLibrary         ProcessFilesVBA.bas
 CasseBriques                 Java                 project
 CB                           make_play            QS_10-Hartinger_backup.c
'ChiZeta Default CTors.cpp'   Mockepon             QS_10-Hartinger.c
 CircleAreaApprox.c           Mockepon_old_stuff   tmp
 debug-out.c                  module.c             vector2D.c
 definitions.h                myProgram            vector2D.h
 findroot_sqrt.c              myProgram.c          VivadoProjects
 freeglut_test.c              Playground
Inhalt des aktuellen Arbeitsverzeichnisses:ls schließt ab mit Errorlevel 0
gcc: error: progDoesNotExist.c: Datei oder Verzeichnis nicht gefunden
gcc: fatal error: no input files
compilation terminated.
Versuch der Kompilation von 'progDoesNotExist.c':Errorlevel 256
\end{minted}
\end{cmdbox}


\subsection{Fehlerbeschreibung ermitteln: \texttt{strerror}}
Niemand will sich Fehlercodes im Kopf behalten. Um eine sinnvolle Unterstüzung bei der Fehlersuche zu geben, existiert die Funktion \texttt{strerror} aus dem Header \texttt{<string.h>}. Dieser wird ein Fehlercode übergeben; der Rückgabewert ist ein Pointer auf einen String, der diesen Fehler menschenlesbar beschreibt:

\begin{codebox}[Beispiel: Errorcodes mit \texttt{strerror} beschreiben]
\begin{minted}[linenos]{c}
#include <stdio.h>
#include <stdlib.h>
#include <string.h>
 
int main() {
  printf("Versuch der Ausführung von 'progDoesNotExist':");
  int errCode_pdne = system("gcc progDoesNotExist.c");
  printf("Errorlevel %d\n", errCode_pdne);  
  printf("Fehlerbeschreibung: %s\n", strerror(errCode_pdne));
  
  return 0;
}
\end{minted}
\end{codebox}

\section{Kommandozeilenparameter} \label{sec:cmdlineParams}
Dem \texttt{gcc} können wir über die Kommandozeile mitteilen, welche Codes wir kompilieren wollen. Außerdem ist es möglich, ober Kommandozeilenparameter wie \texttt{-std=c11} das Verhalten des Compilers weiter zu beeinflussen. Für unsere eigenen Programme gibt es ebenfalls Möglichkeiten, auf solche Kommandozeilenparameter zu reagieren.

Wird ein Programm von der Konsole aus gestartet, so zerlegt das Betriebssystem die komplette Zeile in ihre Bestandteile und kopiert diese in den Arbeitsspeicher\footnote{Vermutlich sind Sie gewohnt, Programme über ein graphisches Userinterface zu starten, also durch Anklicken. Tatsächlich erzeugen diese Maus-Aktionen \enquote{unter der Oberfläche} ebenfalls Kommandozeilen-Befehle.}. Zerlegt wird an denselben Trennzeichen, die auch von \texttt{scanf} akzeptiert werden: Leerzeichen und Tabulatoren. Ein Programmaufruf wie
\begin{cmdbox}[Programmaufruf mit Kommandozeilenparametern]
./myProg parameter1, parameter2        parameter3 "parameter4 with whitespaces"
\end{cmdbox}
wird damit zerlegt in:
\begin{cmdbox}[Zerlegung des Programmaufrufs]
./myProg\\
parameter1,\\
parameter2 \\
parameter3 \\
parameter4 with whitespaces
\end{cmdbox}

Wie Sie sehen, bleiben Kommata erhalten; Leerzeichen werden komplett aus der Zerlegung entfernt, es sei denn, sie befinden sich in \texttt{"}doppelten Anführungszeichen\texttt{"}.

Diese Zerlegung wird unserem Programm zugänglich gemacht, indem wir die Signatur der \texttt{main} abändern:
\begin{codebox}[Syntax: \texttt{main} mit Zugriff auf Kommandozeilenparameter]
\begin{minted}{c}
int main (int argc, char ** argv) {...
\end{minted}
\end{codebox}

Das Symbol \texttt{argc}\footnote{Technisch spricht nichts dagegen, andere Symbole als \texttt{argc} und \texttt{argv} zu verwenden, solange die Signatur einen \mintinline{c}{int}- und einen \mintinline{c}{char **}-Parameter beschreibt. Es hat sich jedoch etabliert, genau diese Namen zu vergeben und sollte im Sinne des Austauschs mit anderen ProgrammiererInnen so beibehalten werden.} (kurz für \emph{argument count}) enthält die Anzahl der \enquote{Blöcke} dieser Zerlegung -- im obigen Beispiel also den Wert \texttt{5}. Über \texttt{argv} (kurz für \emph{argument value}) kann auf die einzelnen Blöcke selbst zugegriffen werden. Machen Sie sich hierzu bitte zuerst den Datentyp \mintinline{c}{char **} klar: Die Zerlegung des Kommandozeilenaufruf ist eine \emph{Liste von Strings}. Ein String wiederum ist eine \emph{Liste von Zeichen}. Damit ist die Zerlegung eine \emph{Liste von Listen von Zeichen} -- ein Pointer zweiter Ebene, also ein \mintinline{c}{char **}. Mit diesen Symbolen können wir nun Code schreiben, und beispielsweise mit \mintinline{c}{if} und \texttt{strcmp} auf bestimmte Parameter reagieren.

Die oben gezeigte Zerlegung wurde mit folgendem Code ausgegeben:
\begin{codebox}[Beispiel: Abfrage von Kommandozeilenparametern]
\begin{minted}[linenos]{c}
#include <stdio.h>

int main(int argc, char ** argv) {
  printf("argc: %d\n", argc);
  
  for (int i=0; i<argc; i++) {printf("%02d\t%s\n", i, argv[i]);}
  
  return 0;
}
\end{minted}
\end{codebox}

\begin{hintbox}[Wert von \texttt{argv[0]}]
Achtung: Der \enquote{nullte} Parameter \texttt{argv[0]} existiert immer und enthält den Namen der ausführbaren Datei, die gerade läuft.
Ein schneller Weg zu prüfen, ob überhaupt Parameter übergeben wurden ist der Vergleich \texttt{argc > 1}.
\end{hintbox}

\section{Spezielle Befehle für unixoide Konsolen}
Die verschiedenen Konsolen-Programme, die auf unixoiden\footnote{\ie. Linux, Mac, BSD, \ldots}  Systemen verbreitet sind können bestimmte \enquote{Kommandos} interpretieren, die ihnen mit einem regulären Druck-Befehl mitgeteilt werden. Das bedeutet, dass wir im Format-String von \texttt{printf} Zeichenketten setzen können, die besondere Effekte auf der Konsole haben.

\subsection{Konsolenbildschirm leeren}
Um einen \enquote{leeren Bildschirm} anzuzeigen, kann die Sequenz \texttt{\textbackslash 033[H\textbackslash 033[J} gedruckt werden.

\begin{codebox}[Beispiel: Abfrage von Kommandozeilenparametern]
\begin{minted}[linenos]{c}
#include <stdio.h>

int main() {
  char password[20];
  
  printf("Bitte Passwort eingeben:\n");
  scanf("%19s", password);
  
  // Bildschirm leeren, um Passwort nicht unnötig lange anzuzeigen
  printf("\033[H\033[J");
  
  return 0;
}
\end{minted}
{\normalfont Anmerkung: Benutzen Sie in der Praxis nie diese Methode, um Passwörter zu verbergen!}
\end{codebox}

\begin{hintbox}[Alternative Wege]
Sowohl unter unixoiden Systemen als auch unter Windows existiert ein Systemkommando, das denselben Effekt hat. Der Aufruf hierfür ist
\begin{center}
	\mintinline{c}{system("clear");}
\end{center}
für unixoiden Systeme und
\begin{center}
	\mintinline{c}{system("cls");}
\end{center}
für Windows.
\end{hintbox}

\subsection{Cursor auf der Konsole platzieren}
Bisher konnten wir auf der Konsole nur \enquote{von oben nach unten} und \enquote{von links nach rechts} schreiben. Dies erfordert manchmal einige Klimmzüge, ein bestimmtes Bild der Darstellung zu erzeugen. Leichter wird diese Aufgabe, wenn wir den Cursor an beliebigen Koordinaten platziern können. Dies ermöglicht die Sequenz \texttt{\textbackslash 033[<Zeile>;<Spalte>H}, wobei \texttt{<Zeile>} und \texttt{<Spalte>} jeweils als Dezimalzahlen angegeben werden.

\begin{codebox}[Beispiel: Sternchen in Tabelle setzen]
\begin{minted}[linenos]{c}
#include <stdio.h>

void print_at(char * text, int row, int col) {
  printf("\033[%d;%dH", (row), (col));
  printf("%s", text);
}

int main () {
  // Tabelle zeichnen
  for (int row = 0; row < 7; row++) {
    if (row % 2) {
      printf("| | | |\n");
    } else {
      printf("+-+-+-+\n");
    }
  }
  
  print_at("*", 4, 4);   // Koordinaten der mittleren Zelle
  print_at("" , 8, 1);   // Cursorposition auf "Ende" setzen
  
  return 0;
}
\end{minted}
\end{codebox}

\begin{cmdbox}[Ausgabebeispiel: Sternchen in Tabelle setzen]
\begin{minted}{text}
+-+-+-+
| | | |
+-+-+-+
| |*| |
+-+-+-+
| | | |
+-+-+-+
\end{minted}
\end{cmdbox}

\subsection{Farben setzen}
Der Cursor hat eine \emph{Schriftfarbe}, die sich aus einer \emph{Vordergrund-} und einer \emph{Hintergrundfarbe} zusammensetzt. Das bedeutet, dass jedes Zeichen in einer aktuell eingestellten Farbe gedruckt wird. Diese Farbe ist standardmäßig grau auf schwarz; wir können jedoch über bestimmte Sequenzen diese Voreinstellungen ändern. Diese Sequenzen sind im Anhang in Tabelle \ref{tab:bashFormatCol} aufgelistet. Hinzu kommen einige \emph{Schrifteffekte} wie Fettdruck, Unterstreichung oder Blinken, die jedoch nicht von allen Konsolenprogrammen unterstützt werden. Die entsprechenden Sequenzen finden Sie in Tabelle \ref{tab:bashFormatSpc}.

\begin{codebox}[Beispiel: Farbige Textausgabe]
\begin{minted}[linenos]{c}
#include <stdio.h>

int main () {
  printf("normaler Text\n");
  
  printf("\x1b[92m");  // Schriftfarbe Hellgrün
  printf("Hellgrün auf schwarz\n");
  printf("Immer noch Hellgrün auf Schwarz\n");
  
  printf("\x1b[44m");  // Hintergrundfarbe Blau
  printf("Hellgrün auf blau\n");
  
  return 0;
}
\end{minted}
\end{codebox}

\begin{cmdbox}[Ausgabebeispiel: Farbige Textausgabe]
normaler Text\\
\textcolor{green}{Hellgrün auf schwarz}\\
\textcolor{green}{Immer noch Hellgrün auf Schwarz}\\
\colorbox {blue} {\textcolor{green}{Hellgrün auf blau}}
\end{cmdbox}

Nachdem die Konsolenfarbe gesetzt wurde, halten sich \emph{alle} Druck-Befehle an die zuletzt gesetzten Schrift\emph{attribute}. Das bedeutet, dass insbesondere auch die Sequenz \emph{Konsolenbildschirm leeren} keinen schwarzen Bildschirm erzeugt, sondern mit der gesetzten Hintergrundfarbe den Bildschirm ausfüllt.

Die genannten Farbnamen sind Standardeinstellungen der üblichen Konsolen-Programme, können jedoch in den Einstellungen leicht geändert werden. Gegebenenfalls stellen Sie mit den genannten Sequenzen also andere Farben ein als Sie erwarten.

\begin{hintbox}[Farbe setzen unter Windows]
Unter Windows existiert der Konsolenbefehl \texttt{color}, dem eine zweistellige Hexadezimalzahl als Parameter mitgegeben wird. Die höherwertige Hexadezimalziffer stellt die Hintergrundfarbe ein, während das zweite Zeichen die Vordergrundfarbe bestimmt. Die übliche Zuordnung von Ziffern zu Farben ist:
\begin{center}
\begin{tabular}{cc|cc||cc|cc}
	0 & Schwarz	& 4 & Rot		&	8 & Dunkelgrau	& C & Hellrot \\
	1 & Blau		& 5 & Magenta	&	9 &	Hellblau		& D & Pink    \\
	2 & Grün		& 6 & Braun		&	A &	Hellgrün		& E & Gelb    \\
	3 & Cyan		& 7 & Grau		&	B & Hellcyan		& F & Weiß
\end{tabular}
\end{center}
Damit bewirkt zum Beispiel unter Windows die C-Code-Zeile:
\begin{center}
	\mintinline{c}{system("color 0a");}
\end{center}
dass die Schriftfarbe \enquote{Hellgrün auf Schwarz} eingestellt wude. Diese Einstellung gilt wie auch unter unixoiden Systemen solange, bis eine neue Farbe eingestellt wird. Weitere Attribute wie Fettdruck sind unter Windows-Konsolen nicht möglich.
\end{hintbox}